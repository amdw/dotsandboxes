\documentclass[a4paper,twocolumn]{article}

\usepackage{palatino}
\usepackage{amsthm}

\newtheorem{thm}{Theorem}[section]
\newtheorem{sstealing}[thm]{Theorem}
\newtheorem{loonybad}[thm]{Corollary}
\newtheorem{freecoins}[thm]{Theorem}

\begin{document}

\title{The Dots-and-Boxes Game}
\author{Andrew Medworth (\texttt{https://github.com/amdw})}
\date{\today}
\maketitle

\begin{abstract}
  An introduction to the game dots-and-boxes, and the more general game
  strings-and-coins. Aims to explain why the game is interesting, and discuss
  some strategic techniques, giving mathematical proofs where appropriate.
\end{abstract}

\tableofcontents

\section{Introduction}

\subsection{The rules of dots-and-boxes}

Dots-and-boxes is a game for two players.

\begin{enumerate}
  \item The game is played on a rectangular grid of dots.
  \item The players take it in turns to draw a line of their choice
    connecting a pair of horizontally or vertically adjacent dots.
  \item If a player completes a $1 \times 1$ box, she scores a point
    (normally recorded by writing her initial inside it) and
    \emph{must} make another move: a player's turn only comes to an
    end when she makes a move which does not complete a box.
  \item The game ends when all boxes have been completed; the winner
    is the player who has completed more boxes.
\end{enumerate}

An example game is shown in Figure TODO.

The rules of dots-and-boxes are very simple: the rules can be learned in
moments, and it can be played purely with pencil and paper. The size of the
board can be varied to alter the simplicity and length of the game: the board
with $5 \times 5$ boxes is already quite deep. The strategy of the game is quite
complex and interesting, and the aim of this paper is to explore it in more
detail.

\subsection{Strings-and-coins}

Dots-and-boxes is a special case of a more general game called
\emph{strings-and-coins}.

\begin{enumerate}
  \item This game begins with a set of coins, and a set of strings
    each connecting a pair of coins; some coins also have one or more
    strings connecting them to the ground.
  \item The players take it in turns to remove one string of their
    choice.
  \item If a player removes the last string connected to a coin, she
    wins the coin, removing it from the game, and \emph{must} play
    another move, her turn ending only when she removes a string which
    does not win any coins.
  \item The game ends when all coins are gone; the winner is the
    player who has captured more coins.
\end{enumerate}

The $m \times n$ game of dots-and-boxes is the special case of
strings-and-coins where $mn$ coins are connected in a rectangular $m
\times n$ grid, with the outer coins connected to the ground. The
strings-and-coins game parallel to the dots-and-boxes game in Figure
TODO is shown in Figure TODO; connections to the ground are indicated
by arrows.

Strings-and-coins is a more general game, because it is possible to
start with a graph (indeed, a multigraph, since the same pair of coins
can be joined by more than one string) of any shape and connectivity
whatsoever.

Anything we learn about strings-and-coins will also apply to
dots-and-boxes; we will therefore mostly focus our attention on
strings-and-coins, occasionally making reference to dots-and-boxes
where something we learn has specific implications there.

\section{Definitions and basic strategy}

\subsection{Basic terminology}

We will call the two players $A$ and $B$, with $A$ generally moving
first in a given position.

The \emph{valency} or \emph{degree} of a coin is the number of strings
connected to it (these terms are borrowed from graph theory). In the
starting position of dots-and-boxes, all coins have valency 4; each
move reduces the valency of either one or two coins by 1. A coin of
valency 1 can be immediately captured.

A \emph{joint} is a coin of degree 3 or more.

A \emph{loop} is a connected set of coins which all have valency 2, and
form a circuit. An $n$-loop is a loop consisting of $n$ coins. (In
dots-and-boxes, there are no loops with fewer than 4 coins, and due to
the symmetry, all loops have an even number of coins.)

A \emph{(closed) chain} is a connected set of coins with valency 2
which is not a loop. A chain has two ends, which can either be
connected to the ground or to a joint. An \emph{open chain} is one
which can be immediately captured, i.e. a chain where one or both ends
has valency 1. A $n$-chain is a chain consisting of $n$ coins.

The act of \emph{opening} a chain or loop is cutting it so that it
becomes an open chain, which can then be captured.

A \emph{double-cross move} is a move which wins two coins at once
(because the string removed connected two coins of valency 1). These
moves are of strategic importance for reasons we will see later.

Let $V(P)$ be the net value of a position $P$, assuming optimal play
from both sides, from the perspective of the player who has to play
first in it. Clearly $V(P)$ will be an integer, and could be either
positive, zero or negative, depending on whether $P$ is advantageous
for the first player. $V(P)$ is also known as the \emph{minimax} value
of $P$ and $V$ is the \emph{value function} of strings-and-coins.

\subsection{Strategic foundations}

To understand the strategy of a game, it is often a good idea to begin
with positions close to the end of the game, since the understanding
of earlier positions depends on later ones.

Let us consider a very simple endgame position: a single closed
$n$-chain. Obviously this position is a 0--$n$ victory for player $B$,
as player $A$ has no alternative but to open the chain for his
opponent, who will take all $n$ coins straight away.

Now consider the position consisting of \emph{two} closed
$n$-chains. At first glance it might appear that this position is a
$n$--$n$ draw, as $A$ must open one chain for $B$, who will take it
and then be forced to open the other one for $A$. However this is not
true: when $A$ opens the first chain, $B$ can take all but the last
two boxes, and then sacrifice them with a \emph{double-dealing move}
as shown in Figure TODO. Because this move does not win a coin, it
ends $B$'s turn, and forces $A$ to make a move. $A$ might as well take
the two sacrificed coins, because in either case, he is again forced
to open the remaining chain for $B$, which $B$ then wins. Thus $B$
wins by $2n - 2$ points to 2.

Notice that this only works with $n \ge 3$: a closed chain of length 2
can be opened by removing the middle string, in which case a
double-dealing move is impossible. Only when $n \ge 3$ is there no way
to open a closed $n$-chain without allowing a double-dealing move.

A similar situation arises with loops, but with one important
difference: when a loop is opened, in order to play a double-dealing
move a player must sacrifice 4 boxes, rather than 2, as shown in
Figure TODO. This is because an opened loop has no link to the ground,
so the only way to avoid taking a coin at the end (and thus be forced
to continue making moves) is to take the string between two connected
pairs of coins, sacrificing both pairs. This means that a
double-dealing move is only possible with an $n$-loop if $n \ge 4$.

Double-dealing moves are an absolutely central concept in
strings-and-coins strategy, because they allow a player to maintain
\emph{control} of a position, to decide whether to play first in the
rest of the position or to force the opponent to do so.

Because of this, a \emph{long chain} is defined as a chain of at least
\emph{three} coins, and a \emph{long loop} is defined as a loop of at
least \emph{four} coins. A long chain or loop is one which cannot be
opened without allowing a double-dealing move.

Further, a \emph{loony move} is defined as a move which allows the
opponent to play a double-dealing move. There are three types of loony
move:

\begin{enumerate}
  \item Opening a long chain
  \item Opening a long loop
  \item Opening a 2-chain by cutting one of the links to the ground:
    this is called a \emph{half-hearted handout}, in contrast to a
    \emph{hard-hearted handout} which is opening the 2-chain by
    cutting the middle link (and is not a loony move).
\end{enumerate}

When a player makes a loony move, it hands over control of the
position to the opponent, and so if a player can force her opponent to
make a loony move, she is almost always winning. (We will discuss the
exceptions shortly.)

These central concepts of strings-and-coins strategy bring us to our
first theorem.

\begin{sstealing}
  If a player has just made a loony move resulting in position $L$,
  her opponent can score at least half of the remaining points,
  regardless of the rest of the position: in other words, $V(L) \ge
  0$.
\end{sstealing}

\begin{proof}
  Let $P$ be the position with the coins offered by the loony move
  removed.
  
  The proof uses a technique called \emph{strategy-stealing}. It
  relies on the fact that from $L$, the player to move can choose
  whether to take all the coins on offer and play first in $P$, or
  play a double-dealing move, sacrificing a few coins but forcing the
  opponent to play first in $P$.

  Let $s$ be the minimum number of coins the player would have to
  sacrifice in order to play a double-dealing move, and let $m$ be the
  total number of coins placed on offer by the loony move. Note that
  $s$ is either 2 or 4, depending on whether a chain or a loop has
  been opened, and $m \ge s$.

  By taking all the coins and playing first in $P$, the player's
  outcome would be $m + V(P)$. By taking all but $s$ coins and forcing
  the opponent to play first in $P$, the player's outcome would be
  $m-2s-V(P)$. So $$V(L) = \max(m+V(P), m-2s-V(P))$$

  If $m+V(P) \ge 0$ then $V(L) \ge 0$ straight away, so suppose
  $m+V(P) < 0$. Then as $m \ge s$, $$m-2s-V(P) > 2m-2s \ge 0$$ so in
  this case $V(L) \ge 0$ also.
\end{proof}

\begin{loonybad}
  A loony move is never better than a non-loony move in the same position.
\end{loonybad}
\begin{proof}
  TODO
\end{proof}

\begin{freecoins}
  Any coins which can be captured without affecting the ability to
  play a double-dealing move later in the turn should always be
  captured immediately.
\end{freecoins}
\begin{proof}
  TODO
\end{proof}

3-chains and loops make things complicated as control doesn't always win.

\begin{thebibliography}{99}
  \bibitem{berl} Elwyn Berlekamp, \emph{The Dots-and-Boxes Game},
    A~K~Peters 2000
\end{thebibliography}

\end{document}
